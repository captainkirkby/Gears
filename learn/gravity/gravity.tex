\documentclass[12pt]{article}
\usepackage{amsmath}
\usepackage{graphicx}

\providecommand{\eqn}[1]{eqn.~(\ref{eqn:#1})}
\providecommand{\tab}[1]{Table~\ref{tab:#1}}
\providecommand{\fig}[1]{Figure~\ref{fig:#1}}

\providecommand{\vecsymbol}[1]{\ensuremath{\boldsymbol{#1}}}
\providecommand{\rv}{\vecsymbol{r}}
\providecommand{\Rv}{\vecsymbol{R}}
\providecommand{\vv}{\vecsymbol{v}}
\providecommand{\Vv}{\vecsymbol{V}}
\providecommand{\mv}{\hat{\vecsymbol{m}}}
\providecommand{\nv}{\hat{\vecsymbol{n}}}
\providecommand{\ev}{\hat{\vecsymbol{e}}}
\providecommand{\thdot}{\dot{\theta}}
\providecommand{\thddot}{\ddot{\theta}}

\title{Gravitational Effects on a Simple Pendulum}
\author{D. Kirkby and D. Kirkby}
\begin{document}

\maketitle

\section{Formalism}

Consider a simple pendulum of fixed length $l$ whose pivot point $\Rv$ is in motion so that the suspended mass $m$ has a position $\rv$ given by
\begin{equation}
\rv(t,\theta) = \Rv(t) + l (\sin\theta \mv + \cos\theta \nv)
\end{equation}
where $\theta$ describes the motion in the plane specified by the unit vectors $\mv$ (``horizontal'') and $\nv$ (``down''). Next, assume that the unit vectors used to define the coordinate $\theta$ are changing with time while maintaining orthonormality $\mv(t)\cdot\mv(t) = 1$, $\nv(t)\cdot\nv(t) = 1$, and $\mv\cdot\nv = 0$, then
\begin{equation}
\rv(t,\theta) = \Rv(t) + l (\sin\theta \mv(t) + \cos\theta \nv(t)) \; .
\end{equation}
We now have a system described by a single generalized coordinate $\theta$ that is constrained by the fixed pendulum length,
\begin{equation}
\left| \rv(t,\theta) - \Rv(t) \right| = l \; ,
\end{equation}
and so that rotation occurs only in the plane defined by $\mv(t)$ and $\nv(t)$. Note that this last constraint is not usually included when analyzing the simple pendulum but is imposed by the leaf spring suspension in a real pendulum clock. Without this constraint, the pendulum plane would rotate (relative to an earth-based reference frame) as is observed with a Foucault pendulum.

The suspended mass has a velocity
\begin{equation}
\vv(t,\theta,\thdot) = \Vv(t) + l\thdot \left[ \cos\theta\,\mv(t) - \sin\theta\,\nv(t) \right]
+ l \left[ \sin\theta\,\dot{\mv}(t) + \cos\theta\,\dot{\nv}(t) \right]
\end{equation}
where
\begin{equation}
\Vv(t) \equiv \frac{d}{dt}\Rv(t)
\end{equation}
and dots denote time derivatives. The corresponding kinetic energy is
\begin{equation}
\begin{split}
T(t,\theta,\thdot) &= \frac{1}{2} m \left( \vv(t)\cdot \vv(t) \right) \\
&= \frac{1}{2} m \left\{ (\Vv(t)\cdot\Vv(t)) + l^2\thdot^2 + 2l^2\thdot (\mv\cdot\dot{\nv}) \right. \\
&\qquad + l^2 \left[\sin^2\theta (\dot{\mv}\cdot\dot{\mv}) + \cos^2\theta (\dot{\nv}\cdot\dot{\nv}) +
2 \sin\theta\cos\theta (\dot{\mv}\cdot\dot{\nv}) \right] \\
&\qquad \left. +2 l\,\Vv(t)\cdot\left[ \cos\theta\,(\thdot \mv+\dot{\mv}) -
\sin\theta\,(\thdot \nv+\dot{\nv})\right] \right\} \; , \\ 
\end{split}
\end{equation}
where we have invoked the orthonormality conditions to simplify this result using
\begin{equation}
\mv\cdot\dot{\mv} = \nv\cdot\dot{\nv} = 0 \quad , \quad
\mv\cdot\dot{\nv} + \dot{\mv}\cdot\nv = 0 \; .
\end{equation}
We also suppress the explicit time dependences of $\mv$, $\nv$, $\dot{\mv}$, $\dot{\nv}$ for notational brevity.

The gravitational potential energy of the suspended mass is given by
\begin{equation}
\begin{split}
V(t,\theta) &= \sum_i \frac{G m M_i}{\left| \Rv_i(t) - \rv(t) \right|} \\
&= \sum_i \frac{G m M_i}{\left|\Delta\Rv_i(t)\right|} \left[
1 + \frac{l^2 - 2 l \Delta\Rv_i(t)\cdot(\sin\theta \mv + \cos\theta \nv)}{\left|\Delta\Rv_i(t)\right|^2}
\right]^{-1/2} \; ,
\end{split}
\end{equation}
with
\begin{equation}
\Delta\Rv_i(t) \equiv \Rv_i(t)-\Rv(t) \; ,
\end{equation}
and where $G$ is Newton's constant and the sum is over the astronomical bodies (earth, moon, sun) contributing to the potential with masses $M_i$ and center-of-mass positions $\Rv_i(t)$. Note that the potential does not depend on $\thdot$. Expanding to first order in $l/|\Delta\Rv_i(t)|$, we find
\begin{equation}
V(t,\theta) \simeq \sum_i \frac{G m M_i}{\left|\Delta\Rv_i(t)\right|} \left[1 +
\frac{l \Delta\Rv_i(t)\cdot(\sin\theta \mv + \cos\theta \nv)}{\left|\Delta\Rv_i(t)\right|^2}\right] \; .
\end{equation}

The pendulum equation of motion is now given by
\begin{equation}
\frac{d}{dt} \left( \frac{\partial L}{\partial \thdot} \right) =
\frac{\partial L}{\partial \theta} \; ,
\end{equation}
with the Lagrangian $L \equiv T - V$. The relevant partial derivatives are
\begin{equation}
\begin{split}
\frac{\partial L}{\partial \thdot}
&= \frac{\partial T}{\partial \thdot} \\
&= m \left\{ l^2\thdot  + l^2 (\mv\cdot\dot{\nv})
+ l\,\Vv(t)\cdot\left( \cos\theta\,\mv - \sin\theta\,\nv\right) \right\}
\end{split}
\end{equation}
and
\begin{equation}
\frac{\partial L}{\partial \theta} = \frac{\partial T}{\partial \theta} - \frac{\partial V}{\partial \theta}
\end{equation}
with
\begin{equation}
\begin{split}
\frac{\partial T}{\partial \theta} &= \frac{m}{2} \left\{
l^2 \left[ \sin 2\theta \left( \dot{\mv}\cdot\dot{\mv} - \dot{\nv}\cdot\dot{\nv}\right) +
\cos 2\theta \left(\dot{\mv}\cdot\dot{\nv}\right) \right] \right. \\
&\qquad \left. -2 l\,\Vv(t)\cdot\left[ \sin\theta\,(\thdot \mv+\dot{\mv}) +
\cos\theta\,(\thdot \nv+\dot{\nv})\right]
\right\}
\end{split}
\end{equation}
and
\begin{equation}
\frac{\partial V}{\partial \theta} = G m 
\sum_i \frac{M_i}{\left|\Delta\Rv_i(t)\right|^3} \,l \Delta\Rv_i(t)\cdot(\cos\theta\, \mv - \sin\theta\, \nv) \; .
\end{equation}
Putting the pieces together and dividing through by $m l$, we find
\begin{multline}
l\thddot + l(\mv\cdot\ddot{\nv} + \dot{\mv}\cdot\dot{\nv}) +
\dot{\Vv}(t)\cdot\left(\cos\theta \mv - \sin\theta \nv\right) \\
- \Vv(t)\cdot \left[ \sin\theta\, (\thdot\mv+\dot{\nv}) + \cos\theta\, (\thdot\nv - \dot{\mv}) \right] =\\
\frac{l}{2} \left[ \sin 2\theta \left( \dot{\mv}\cdot\dot{\mv} - \dot{\nv}\cdot\dot{\nv}\right) +
\cos 2\theta \left(\dot{\mv}\cdot\dot{\nv}\right) \right]\\
-\Vv(t)\cdot\left[ \sin\theta\,(\thdot \mv+\dot{\mv}) +
\cos\theta\,(\thdot \nv+\dot{\nv})\right]\\
-G \sum_i \frac{M_i}{\left|\Delta\Rv_i(t)\right|^3} \Delta\Rv_i(t)\cdot(\cos\theta\, \mv - \sin\theta\, \nv) \; .
\end{multline}
Combining terms proportional to $\Vv(t)$, we can simplify
\begin{multline}
l\thddot + l(\mv\cdot\ddot{\nv} + \dot{\mv}\cdot\dot{\nv}) +
\dot{\Vv}(t)\cdot\left(\cos\theta \mv - \sin\theta \nv\right) =\\
\frac{l}{2} \left[ \sin 2\theta \left( \dot{\mv}\cdot\dot{\mv} - \dot{\nv}\cdot\dot{\nv}\right) +
\cos 2\theta \left(\dot{\mv}\cdot\dot{\nv}\right) \right]\\
-\Vv(t)\cdot\left[ \sin\theta\,(\dot{\mv}-\dot{\nv}) +
\cos\theta\,(\dot{\mv}+\dot{\nv})\right]\\
-G \sum_i \frac{M_i}{\left|\Delta\Rv_i(t)\right|^3} \Delta\Rv_i(t)\cdot(\cos\theta\, \mv - \sin\theta\, \nv) \; .
\end{multline}

For the plane of rotation used to define $\theta$, we take $\nv$ to point from the pendulum pivot towards the Earth's center of gravity $\Rv_e$
\begin{equation}
\nv(t) = \frac{\Rv_e(t) - \Rv(t)}{\left|\Rv_e(t) - \Rv(t)\right|} \; .
\end{equation}
We then define unit vectors $\ev_1$ and $\ev_2$ that establish the easterly and northerly directions at $\Rv$, respectively, as
\begin{equation}
\ev_1(t) = \frac{\Vv(t) - \Vv_e(t)}{\left|\Vv(t) - \Vv_e(t)\right|} \quad , \quad
\ev_2(t) = \nv(t) \times \ev_1(t) \; .
\end{equation}
Note that $\nv(t)$ and $\ev_1(t)$ are orthogonal since the earth rotates as a rigid body, and therefore $\nv$, $\ev_1$, $\ev_2$ define a right-handed coordinate system. We assume that the pendulum's plane of rotation is constrained to a plane at a fixed orientation in the rotating $\ev_1$-$\ev_2$ plane
\begin{equation}
\mv(t) = \sin\alpha\,\ev_1 + \cos\alpha\,\ev_2
\end{equation}
where $\alpha$ is the usual azimuthal compass bearing angle ($\alpha = 0^\circ, 90^\circ, 180^\circ, 270^\circ$ point north, east, south, west).

\end{document}
